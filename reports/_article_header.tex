\documentclass[12pt]{article}
%\documentclass[reqno]{amsart}
%\documentclass[titlepage]{amsart}

%% {{{{
%% This package is already loaded by beamer
% https://tex.stackexchange.com/questions/314344/beamer-presentation-compile-error
\usepackage{graphicx}
%% }}}}

%http://tex.stackexchange.com/questions/36797/how-can-i-make-todonotes-use-all-of-the-margin
\usepackage{fullpage}
%\usepackage{showframe}
% \usepackage[paperwidth=210mm,
%             paperheight=297mm,
%             left=50pt,
%             top=50pt,
%             textwidth=345pt,
%             marginparsep=25pt,
%             marginparwidth=150pt,
%             textheight=692pt,
%             footskip=50pt]
%            {geometry}

%% {{{{
% https://tex.stackexchange.com/questions/9796/how-to-add-todo-notes
\usepackage{xargs} % Use more than one optional parameter in a new commands 
\usepackage[dvipsnames]{xcolor}
% \newcommandx{\todoproposal}[2][1=]{\todo[linecolor=Plum,backgroundcolor=Plum!25,bordercolor=Plum,#1]{#2}}
\newcommandx{\todoproposal}[2][1=]{\todo[disable, linecolor=Plum,backgroundcolor=Plum!25,bordercolor=Plum,#1]{#2}}
% \newcommandx{\tododraft}[2][1=]{\todo[#1]{#2}}
\newcommandx{\tododraft}[2][1=]{\todo[disable, #1]{#2}}
% \newcommandx{\thiswillnotshow}[2][1=]{\todo[disable,#1]{#2}}
%% Math environment in todo note
%
% https://tex.stackexchange.com/questions/298404/todonotes-and-reserveda-nested-itemize-enumerate-environments/298405#298405
\newcommand\todoin[2][]{\todo[inline, caption={2do}, #1]{
\begin{minipage}{\textwidth-4pt}#2\end{minipage}}}
% \newcommand\todoin[2][]{\todo[disable, inline, caption={2do}, #1]{
% \begin{minipage}{\textwidth-4pt}#2\end{minipage}}}
%% }}}}


%% {{{{
%http://tex.stackexchange.com/questions/44858/adding-the-word-appendix-to-table-of-contents-in-latex
\usepackage[titletoc, page]{appendix}
%% }}}}

%% {{{{
	
%I'm using this package to put todo notes into my document. Then, I can
%use it to put a list of the todo notes at the end of the document.
\usepackage[textsize=footnotesize]{todonotes}
% \usepackage[disable=true, colorinlistoftodos,prependcaption,textsize=footnotesize]{todonotes}
%This package has some conficts with amsart. To resolve this, I use
%the following code.
\makeatletter
\providecommand\@dotsep{5}
\def\listtodoname{List of Todos}
\def\listoftodos{\@starttoc{tdo}\listtodoname}
\makeatother
%I got this workaround code from the package's documentation:
%http://get-software.net/macros/latex/contrib/todonotes/todonotes.pdf

% %\newcounter{chapter}
% %\numberwithin{section}{chapter}
% \theoremstyle{mydefinition}
% \newtheorem{exercise}{Exercise}
% \newcommand{\newproblem}[2]{\setcounter{exercise}{#1}\addtocounter{exercise}{
% -1}\begin{exercise}#2\end{exercise}}
% \newcommand{\setcontext}[2]{\setcounter{chapter}{#1}\setcounter{section}{#2}}

% \newtheorem*{remark}{Remark}

%% }}}}


%% {{{{
% Bibliography as numbered section
% https://tex.stackexchange.com/questions/88890/how-to-get-the-references-section-to-be-numbered-as-if-it-were-created-via-sect
\usepackage[numbib]{tocbibind}
%% }}}}


%%%%%%%%%%%%5
%%%% >>>>

%%%%%%%%%%%%5


%%%%%%%%%%%%%%%%%%%%%%%%
%% Section Styling
%%%%%%%%%%%%%%%%%%%%%%%%%

 \usepackage{titlesec}

% %\titleformat*{\subsection}{\newpage \Large \bfseries}
% \titleformat*{\subsubsection}{\large\itshape}

% \usepackage[explicit]{titlesec}

% %Start section with new page
% %http://tex.stackexchange.com/questions/9497/start-new-page-with-each-section
% \newcommand{\sectionbreak}{\clearpage}

% %Underlining ruler for subsections
% %http://tex.stackexchange.com/questions/84061/how-can-i-make-a-bold-horizontal-rule-under-each-section-title
% \titleformat{\section}
%   {\normalfont\LARGE\bfseries}
%   {
%   \thesection
%   }
%   {1em}
%   {#1}
%   [{\titlerule[0.8pt]}]

% \titleformat{\subsection}
%   {\normalfont\Large\bfseries}
%   {\thesubsection}
%   {1em}
%   {#1}

% \titleformat{\subsubsection}
%   {\normalfont\normalsize\itshape}
%   {\thesubsubsection}
%   {1em}
%   {#1}


% Change format of \paragraph{...}
% http://tex.stackexchange.com/questions/3881/formatting-a-paragraph-to-look-like-a-section
% \titleformat{\paragraph}[hang]{\normalfont\normalsize\itshape}{\theparagraph}{1em}{}
% \titlespacing*{\paragraph}{0pt}{3.25ex plus 1ex minus .2ex}{1em}


%%%%%%%%%%%%%%%%%%%%%%%%%%%%%%%%%%%%%%%%%%%%%%
%% Change section styling for HW documents %%
%%%%%%%%%%%%%%%%%%%%%%%%%%%%%%%%%%%%%%%%%%%%%%

%%%
% This first method uses the base LaTeX package:
% http://tex.stackexchange.com/questions/85011/section-and-subsection-heading-style

%\def\@seccntformat#1{\csname the#1\endcsname\quad} %default

%\def\@seccntformat#1{Problem \csname the#1\endcsname\quad} 

% %%% 
% % This next one uses the `titlesec` packages. The references I used are
% % here:
% % http://tex.stackexchange.com/questions/140447/changing-section-heading-style
% % http://tex.stackexchange.com/questions/37189/number-subsections-and-subsubsections-but-not-sections
% \usepackage{bookmark}
% \usepackage{titlesec}
% \titleformat{\section}
%   {\normalfont\bfseries\Large}{Problem \thesection}{1em}{}
% \titleformat{\subsection}
%   {\normalfont\large\itshape}{\thesubsection}{1em}{}
% \titleformat{\subsubsection}
%   {\normalfont\normalsize\itshape}{\thesubsubsection}{1em}{}

% % Add proper labels to PDF bookmarks.
% % http://tex.stackexchange.com/questions/156530/how-to-change-the-pdfbookmark-titles-hyperref
% \makeatletter
% \bookmarksetup{%
%   addtohook={%
%     \ifnum\toclevel@section=\bookmarkget{level}\relax
%       \renewcommand*{\numberline}[1]{Problem #1 }%
%     \fi
%   },
% }



%Option to number subsection with letters
% http://tex.stackexchange.com/questions/74529/sections-indexed-with-numbers-subsections-with-letters
% \renewcommand{\thesection}{\Alph{section}}
% \renewcommand{\thesubsubsection}{\thesubsection.\alph{subsubsection}}
% %%%%%%%%%%%%%%%%%%%%%%%%%%%%%%%%%%%%%%%%%%%%%

%% {{{{
%Links, esp. from table of contents
% http://timmurphy.org/2014/03/11/latex-table-of-contents-with-clickable-links/
\usepackage{hyperref}
\hypersetup{
    colorlinks=true, % make the links colored
    linkcolor=blue, % color TOC links in blue
    urlcolor=red, % color URLs in red
    linktoc=all, % 'all' will create links for everything in the TOC
    %citecolor=gray
    citecolor=blue
    }
%% }}}}

\usepackage{setspace}