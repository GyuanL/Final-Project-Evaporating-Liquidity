\documentclass{article}

\usepackage[english]{babel}
\usepackage[letterpaper,top=2cm,bottom=2cm,left=3cm,right=3cm,marginparwidth=1.75cm]{geometry}
\usepackage{amsmath}
\usepackage{booktabs}
\usepackage{caption}
\usepackage{graphicx}
\usepackage{makecell}
\usepackage{pdflscape}


\newcommand*{\PathToOutput}{../output/}


\begin{document}

\title{Evaporating Liquidity - Replication Report}

\author{
    Ruilong Guo,
    Sifei Zhao,
    Zhiyuan Liu,
    Junhan Fu
}


\maketitle


\section{Introduction}
This project replicates Table 1 and 2 in Evaporating Liquidity\cite{nagel}. The author
shows that the returns of short-term reversal strategies are generated by liquidity 
provision, and therefore are highly predictable by the VIX index. The author also 
found that reversal strategies on not only individual stocks but also industry portfolios 
produce high returns, especially during periods of high VIX.

The author constructs the reversal strategy by averaging the returns of five substrategies
that weight stocks (or industries) proportional to the negative of market-adjusted returns
on days $t-1$ to $t-5$.
\begin{equation}
    w_{it}^R = -\left( \frac{1}{2} \sum_{i=1}^{N} \left| R_{it-1} - R_{mt-1} \right| \right)^{-1} \left( R_{it-1} - R_{mt-1} \right),
\end{equation}
where $R_{mt-1} = \frac{1}{N}\sum_{i=1}^N R_{it-1}$ is the equal-weighted market return.
Table 1 reports the summary statistics of the reversal strategies on individual stocks 
and industry portfolios. For individual stocks, the returns are calculated based on 
end-of-day transaction prices and quote midpoints.

Table 2 reports the results of the following predictive regression
\begin{equation}
    L_t^R = a + bVIX_{t-5} + c'g_{t-5} + e_t,
\end{equation}
where $L_t^R$ is the return of the reversal strategy. $VIX_{t-5}$ is the VIX index lagged
by 5 days, divided by $\sqrt{250}$. $g_{t-5}$ is a vector of control variables, including 
pre-decimalization dummy (takes a value of one prior to April 9, 2001 and a value of zero 
thereafter) and market return.

This project replicates these two tables using the same sample range as the original
paper (from January 1998 to December 2010). We also provide the updated tables using
data from January 1998 to December 2023.
\section{Data Description}


\section{Replicated and updated results}

\begin{landscape}
\begin{table}
    \centering
    \caption*{Table 1:  Summary Statistics of Reversal Strategy Returns}
\end{table}


\begin{table}
    \centering
    \caption*{Table 1:  Summary Statistics of Reversal Strategy Returns (Replicated)}
\end{table}


\begin{table}
    \centering
    \caption*{Table 1:  Summary Statistics of Reversal Strategy Returns (Updated)}
\end{table}

\end{landscape}


\begin{landscape}
\begin{table}
    \centering
    \caption*{Table 2: Predicting Reversal Strategy Returns with VIX}
    
    \small Original Table 2 from the paper.
    \medskip

    \begin{tabular}{lcccccccccccc}
\toprule
& \multicolumn{4}{c}{\makecell{Individual stocks\\Transaction-price returns}} & \multicolumn{4}{c}{\makecell{Individual stocks\\Quote-midpoint returns}} & \multicolumn{4}{c}{\makecell{Industry\\portfolios}} \\
& \multicolumn{3}{c}{Daily} & Monthly & \multicolumn{3}{c}{Daily} & Monthly & \multicolumn{3}{c}{Daily} & Monthly \\
& (1) & (2) & (3) & (4) & (5) & (6) & (7) & (8) & (9) & (10) & (11) & (12) \\
\midrule
Intercept & -0.03 & -0.05 & -0.02 & 0.02 & -0.06 & -0.07 & -0.04 & -0.01 & -0.08 & -0.09 & -0.06 & -0.05 \\
& (0.03) & (0.02) & (0.02) & (0.02) & (0.03) & (0.03) & (0.03) & (0.02) & (0.02) & (0.02) & (0.02) & (0.01) \\
VIX & 0.22 & 0.20 & 0.18 & 0.15 & 0.16 & 0.16 & 0.13 & 0.10 & 0.07 & 0.07 & 0.05 & 0.04 \\
& (0.02) & (0.02) & (0.02) & (0.01) & (0.02) & (0.02) & (0.02) & (0.02) & (0.02) & (0.02) & (0.02) & (0.01) \\
Pre-decim. & & 0.22 & 0.22 & 0.23 & & 0.08 & 0.09 & 0.09 & & 0.00 & 0.01 & 0.01 \\
& & (0.03) & (0.03) & (0.03) & & (0.03) & (0.03) & (0.03) & & (0.02) & (0.02) & (0.02) \\
$R_M$ & & & -0.60 & -0.03 & & & -0.59 & -0.16 & & & -0.42 & -0.05 \\
& & & (0.19) & (0.26) & & & (0.21) & (0.28) & & & (0.17) & (0.16) \\
Adj. $R^2$ & 0.07 & 0.11 & 0.11 & 0.56 & 0.03 & 0.03 & 0.04 & 0.25 & 0.01 & 0.01 & 0.01 & 0.07 \\
\bottomrule
\end{tabular}

\end{table}


\begin{table}
    \centering
    \caption*{Table 2: Predicting Reversal Strategy Returns with VIX (Replicated)}

    \raggedright
    \small Replicated Table 2, which uses the same sample range as the original (from 
    January 1998 to December 2010). 
    It has been verified that coefficients of predictor variables in the replicated result
    have the same sign with the original result. The coefficients of replicated result are
    within the 99.7\% confidence interval of the original result.
    \medskip

    \centering
    \begin{tabular}{lcccccccccccc}
\toprule
 & \multicolumn{4}{c}{\makecell{Individual stocks\\Transaction-price returns}} & \multicolumn{4}{c}{\makecell{Individual stocks\\Quote-midpoint returns}} & \multicolumn{4}{c}{\makecell{Industry\\portfolios}} \\
 & \multicolumn{3}{c}{Daily} & Monthly & \multicolumn{3}{c}{Daily} & Monthly & \multicolumn{3}{c}{Daily} & Monthly \\
 & (1) & (2) & (3) & (4) & (5) & (6) & (7) & (8) & (9) & (10) & (11) & (12) \\
\midrule
Intercept & -0.06 & -0.09 & -0.06 & -0.01 & -0.06 & -0.07 & -0.03 & 0.00 & -0.10 & -0.10 & -0.07 & -0.04 \\
 & (0.03) & (0.02) & (0.03) & (0.02) & (0.03) & (0.03) & (0.04) & (0.03) & (0.03) & (0.03) & (0.03) & (0.02) \\
VIX & 0.25 & 0.23 & 0.21 & 0.18 & 0.18 & 0.17 & 0.14 & 0.11 & 0.08 & 0.08 & 0.06 & 0.04 \\
 & (0.02) & (0.02) & (0.02) & (0.01) & (0.03) & (0.03) & (0.03) & (0.02) & (0.02) & (0.02) & (0.02) & (0.01) \\
Pre-decim. &  & 0.23 & 0.24 & 0.25 &  & 0.11 & 0.11 & 0.12 &  & 0.01 & 0.01 & 0.02 \\
 &  & (0.03) & (0.03) & (0.03) &  & (0.03) & (0.03) & (0.03) &  & (0.02) & (0.02) & (0.02) \\
$R_M$ &  &  & -0.45 & 0.10 &  &  & -0.78 & -0.28 &  &  & -0.57 & -0.21 \\
 &  &  & (0.19) & (0.23) &  &  & (0.23) & (0.26) &  &  & (0.21) & (0.16) \\
Adj. $R^2$ & 0.07 & 0.10 & 0.10 & 0.65 & 0.02 & 0.03 & 0.03 & 0.27 & 0.01 & 0.01 & 0.01 & 0.07 \\
\bottomrule
\end{tabular}

\end{table}

\begin{table}
    \centering
    \caption*{Table 2: Predicting Reversal Strategy Returns with VIX (Updated)}

    \small Updated Table 2, using data from January 1998 to December 2023.
    The results are consistent.
    \medskip

    \begin{tabular}{lcccccccccccc}
\toprule
 & \multicolumn{4}{c}{\makecell{Individual stocks\\Transaction-price returns}} & \multicolumn{4}{c}{\makecell{Individual stocks\\Quote-midpoint returns}} & \multicolumn{4}{c}{\makecell{Industry\\portfolios}} \\
 & \multicolumn{3}{c}{Daily} & Monthly & \multicolumn{3}{c}{Daily} & Monthly & \multicolumn{3}{c}{Daily} & Monthly \\
 & (1) & (2) & (3) & (4) & (5) & (6) & (7) & (8) & (9) & (10) & (11) & (12) \\
\midrule
Intercept & -0.08 & -0.08 & -0.05 & -0.01 & -0.09 & -0.09 & -0.06 & -0.02 & -0.09 & -0.09 & -0.07 & -0.06 \\
 & (0.02) & (0.03) & (0.02) & (0.02) & (0.03) & (0.03) & (0.03) & (0.03) & (0.02) & (0.02) & (0.02) & (0.02) \\
VIX & 0.24 & 0.21 & 0.19 & 0.15 & 0.19 & 0.18 & 0.17 & 0.12 & 0.08 & 0.08 & 0.07 & 0.05 \\
 & (0.02) & (0.02) & (0.02) & (0.02) & (0.02) & (0.03) & (0.02) & (0.03) & (0.02) & (0.02) & (0.02) & (0.01) \\
Pre-decim. &  & 0.26 & 0.27 & 0.28 &  & 0.09 & 0.10 & 0.12 &  & 0.00 & 0.00 & 0.01 \\
 &  & (0.03) & (0.03) & (0.03) &  & (0.03) & (0.03) & (0.03) &  & (0.02) & (0.02) & (0.02) \\
$R_M$ &  &  & -0.39 & 0.03 &  &  & -0.47 & -0.04 &  &  & -0.24 & -0.03 \\
 &  &  & (0.17) & (0.18) &  &  & (0.23) & (0.26) &  &  & (0.16) & (0.13) \\
Adj. $R^2$ & 0.04 & 0.05 & 0.05 & 0.53 & 0.02 & 0.02 & 0.02 & 0.19 & 0.01 & 0.01 & 0.01 & 0.08 \\
\bottomrule
\end{tabular}

\end{table}

\end{landscape}

\newpage
\bibliographystyle{jpe}
\bibliography{bibliography.bib}

\end{document}